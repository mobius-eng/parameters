% Created 2016-09-05 Mon 12:40
\documentclass[listings, a4paper, fleqn, pdftex, 12pt, openany, oneside, final]{memoir}
\makeatletter
\usepackage[english]{babel}
\usepackage[T1]{fontenc}
\usepackage{amsmath}
\usepackage{amsthm}
\usepackage{mathtools}
\usepackage{latexsym}
\usepackage{amssymb}
\usepackage{amsfonts}
\usepackage{bm}
\usepackage{fourier}
\usepackage[pdftex]{graphicx}
\usepackage{subfig}
\usepackage{color}
\usepackage{natbib}
\usepackage{hyperref}%[pdftex,colorlinks=true,citecolor=blue]{hyperref}
\usepackage{lscape}
\usepackage{soul}
\usepackage{tabularx}
\usepackage{booktabs}
%\usepackage{listings}
%\usepackage{listing}
\usepackage{minted}
\chapterstyle{verville}
\beforechapskip=-17mm
\textwidth=167mm
\textheight=230mm
\voffset=-10mm
\oddsidemargin=-4mm
\evensidemargin=-4mm
%%%               Vectors                          %%%
\renewcommand{\vec}[1]{\bm{#1}}
\renewcommand{\Vec}[1]{\overline{#1}}
\newcommand{\mat}[1]{\mathbf{#1}}
\newcommand{\gmat}[1]{\vec{#1}}
%%%     Products, derivatives etc.         %%%
\newcommand{\vprod}[2]{\left[#1\times #2\right]}
\newcommand{\sprod}[2]{#1\cdot #2}
\newcommand{\der}[2]{\frac{d #1}{d #2}}
\newcommand{\Der}[2]{\frac{d}{d #2}\left(#1\right)}
\newcommand{\pder}[2]{\frac{\partial #1}{\partial #2}}
\newcommand{\Pder}[2]{\frac{\partial}{\partial #2}\left(#1\right)}
\newcommand{\dder}[2]{\frac{\delta #1}{\delta #2}}
\DeclareMathOperator{\grad}{grad}
\DeclareMathOperator{\Div}{div}
\DeclareMathOperator{\erfc}{erfc}
\DeclareMathOperator{\Expect}{E}
\DeclareMathOperator{\Var}{Var}
\newcommand{\mean}[1]{\langle #1 \rangle}
\newcommand{\const}{\mathop{\mathrm{const}}}
\newcommand{\dimension}[1]{\left[\mathrm{#1}\right]}
%%%     Project specific                         %%%
\newcommand{\Diff}{\mathcal{D}}
%%%%%%%%%%          Alexey's comments                       %%%%%%%%%%
\newcommand{\alemph}[1]{\colorbox{yellow}{#1}}
\newcommand{\red}[1]{{\color{red}#1}}
\newcommand{\alnew}[1]{%\\ul%
}
%%%%%%%%%%    Hypothesis                                          %%%%%%%%%%
\theoremstyle{definition}
\newtheorem{hypothesis}{Hypothesis}
%%%%% to avoid orphans and widows
\widowpenalty3000
\clubpenalty3000
%%% Common Lisp shortcut for minted
\newminted{cl}{}
\makeatother
\author{alexey}
\date{\today}
\title{}
\hypersetup{
 pdfauthor={alexey},
 pdftitle={},
 pdfkeywords={},
 pdfsubject={},
 pdfcreator={Emacs 24.5.1 (Org mode 8.3.5)}, 
 pdflang={English}}
\begin{document}

\tableofcontents

\chapter{Parameter container: complete example}
\label{sec:orgheadline7}
\section{Model}
\label{sec:orgheadline6}
\subsection{Description}
\label{sec:orgheadline1}
This model is based on plug-flow reactor (PFR) example (Levenspiel,
1999; page 106). PFR can be viualised as a tube whith input steam
flowing into one side and output streams coming from another
side. Inside the tube some conversion occurs that alters the
composition of the input stream. The typical characteristic of a
(ideal) PFR are:
\begin{itemize}
\item Conversion is gradual and
\item There is no back-mixing of the flowing stream.
\end{itemize}

These characteristics make it easy to write mass-balance of PFR find
the extent of conversion of the input stream. Conversion is defined as
one minus the ratio of what is coming out over what came in, or more
formally, conversion of the species \(A\) is

\begin{equation}
\label{eq:1}
X_{A} = \left(1 - \frac{F_{A,out}}{F_{A,in}} \right)\times 100\%
\end{equation}
where \(F_{A}\) is molar (or mass) flow rate (in, for example, mol/h). For
bravity, we will not be deriing any mass balance equations for PFR,
but only use a couple of results from Levenspiel, 1999.

PFR is proposed for the decomposition of
phosphine
\begin{equation*}
4\mathrm{PH}_{3g }\to \mathrm{P}_{4(g)} + 6\mathrm{H}_{2}
\end{equation*}
Denote \(A = \mathrm{PH}_{3}\), \(B = \mathrm{P}_{4}\) and \(S =
\mathrm{H}_{2}\). Reaction rate (i.e. the rate at which \(A\) disappears) is
defined using equation:
\begin{equation}
-r_{A}=kC_{A}
\end{equation}
with rate constant \(k\) being dependent on temperature according to
\emph{Arrhenius law}:
\begin{equation}
k=k_{0} e^{-E/R(1/T-1/T_{0})}
\end{equation}

The volume of the reactor required to achieve the conversion \(X_{A}\) is
defined by the equation (Levenspiel, 1999; equation 21):
\begin{equation}
\label{eq:pfr-vol}
V=\frac{F_{A0}}{kC_{A0}}\left[(1+\varepsilon_{A})\ln{\frac{1}{1-X_{A}}} -\varepsilon_{A}X_{A}\right]
\end{equation}
where \(F_{A0}\) is input flow rate of A and \(C_{A0}\) is initial concentration
of A. Constant \(\varepsilon_{A}=(7-4)/4=0.75\) is defined via stoichiometry
(fractional change in volume).

Concentration can be determined from ideal gas law:
\begin{equation}
C_{A0}=\frac{p_{A0}}{RT}
\end{equation}
where \(p_{A0}\) is initial partial pressure of A, \(R\) is universal gas
constant and \(T\) is operating temperature.

This model can be used to estimate the volume of the reactor given
economical constraints. A default parameters are provided as a
guidence but are the subject to change (except for constant \(R\)).

\begin{center}
\begin{tabular}{lrl}
Parameter & Default value & Units\\
\hline
\(k_{0}\) & 10 & 1/h\\
\(E\) & 160 & kJ/mol\\
\(T_{0}\) & 600 & C\\
\(T\) & 650 & C\\
\(F_{A0}\) & 40 & mol/h\\
\(p\) & 4 & atm\\
\(R\) & 8.314 & J/ mol K\\
\end{tabular}
\end{center}


Our goal is to develop computer representation of this model.

\subsection{Reaction rate representation}
\label{sec:orgheadline2}
First, we need to represent the temperature dependence of the rate
constant

\begin{minted}[]{common-lisp}
(defconstant +universal-gas-constant+ 8.314d0
  "Universal gas constant R=8.314 J/mol K")

(defclass arrhenius-law ()
  ((reference-rate-constant
    :initarg :reference-rate-constant
    :accessor reference-rate-constant
    :documentation "Reference rate constant (k0)")
   (activation-energy
    :initarg :activation-energy
    :accessor activation-energy
    :documentation "Activation energy (Ea) in J/mol")
   (reference-temperature
    :initarg :reference-temperature
    :accessor reference-temperature
    :documentation "Reference temperature (T0) in K"))
  (:documentation
   "Arrhenius law of the reaction rate constant:
    k(T) = k0 * exp[- E/R * (1/T - 1/T0)]"))

(defmethod print-object ((object arrhenius-law) out)
  (with-slots ((k0 reference-rate-constant)
               (e activation-energy)
               (t0 reference-temperature))
      object
    (print-unreadable-object (object out :type t)
      (format out "~@<~:_k0 = ~A ~:_Ea = ~A J/mol ~:_T0 = ~A K~:>"
              k0 e t0))))

(defmethod rate-constant ((model arrhenius-law) temperature)
  (with-accessors ((k0 reference-rate-constant)
                   (e activation-energy)
                   (t0 reference-temperature))
      model
    (* k0 (exp (- (* (/ e +universal-gas-constant+) (- (/ temperature) (/ t0))))))))
\end{minted}

\subsection{Ideal gas law}
\label{sec:orgheadline3}
To find the concentration from given pressure and temperature:

\begin{minted}[]{common-lisp}
(defun ideal-gas-concentration (pressure temperature)
  "Calculates concentration (mol/m3) based in PRESSURE (Pa) and
temperature (K) using ideal gas law: c = p / RT"
  (/ pressure (* +universal-gas-constant+ temperature)))
\end{minted}

\subsection{Plug-flow reactor model for phosphine decomposition}
\label{sec:orgheadline4}
Plug flow reactor model is composed of the following parts:
\begin{itemize}
\item Inlet flow rate \(F_{A0}\),
\item Final (target) conversion \(X_{A}\),
\item Operating temperature \(T\),
\item Operating pressure \(p\), and
\item Reaction rate constant defined by Arrhenius law.
\end{itemize}

\begin{minted}[]{common-lisp}
(defclass PFR-phosphine ()
  ((inlet-flow-rate
    :initarg :inlet-flow-rate
    :documentation "F_A0 in mol/s"
    :accessor inlet-flow-rate)
   (final-conversion
    :initarg :final-conversion
    :accessor final-conversion
    :documentation "Target conversion X_A")
   (operating-temperature
    :initarg :operating-temperature
    :accessor operating-temperature
    :documentation "PFR operating temperature (K)")
   (operating-pressure
    :initarg :operating-pressure
    :accessor operating-pressure
    :documentation "PFR operating pressure (Pa)")
   (phosphine-decomposition-rate-coefficient
    :initarg :rate-coefficient
    :accessor phosphine-decomposition-rate-coefficient
    :documentation
    "Rate constant 'k' object specializing RATE-CONSTANT generic function"))
  (:documentation
   "Plug-flow-reactor for phosphine decomposition
    4PH3 -> P4 + 6H2"))

(defmethod print-object ((object PFR-phosphine) out)
  (with-slots ((f inlet-flow-rate)
               (x final-conversion)
               (temperature operating-temperature)
               (p operating-pressure)
               (rate-expression phosphine-decomposition-rate-coefficient))
      object
    (print-unreadable-object (object out :type t)
      (format out "~@<~:_F = ~A mol/s ~:_X = ~A ~:_T = ~A K ~:_p = ~A Pa~:>"
              f x temperature p))))
\end{minted}

Use equation \eqref{eq:pfg-vol} to calculate volume of the reactor:

\begin{minted}[]{common-lisp}
(defmethod PFR-phosphine-volume ((model pfr-phosphine))
  "Uses equation (Levenspiel, 1999; equation 21):
     F   _                          _  
      A0|               1            | 
V = ----| (1 + e )ln{------}  - e X  | 
    kC  |       A    1 - X       A A | 
      A0|_                A         _| 

to calculate the volume of the plug-flow reactor
"
  (with-accessors ((f inlet-flow-rate)
                   (x final-conversion)
                   (temperature operating-temperature)
                   (p operating-pressure)
                   (rate phosphine-decomposition-rate-coefficient))
      model
    (let ((k (rate-constant rate temperature))
          (eps 0.75d0)
          (c (ideal-gas-concentration p temperature)))
      (* (/ f (* k c))
         (- (* (1+ eps) (log (/ (- 1 x))))
            (* eps x))))))
\end{minted}

\subsection{Model parameters}
\label{sec:orgheadline5}
The model consists of a large number of parameters. To complicate the
matter further, some parameters can be represented in different units
of measure. Just to keep the description short, the only different
units parameter we will consider is temperature. To assist in defining
the temperature in different units (Celcius, Kelvin or Fahrenheit) we
will use the following function:

\begin{minted}[]{common-lisp}
(defun make-temperature (default-value default-units name id)
  "Produces PARAMETER-OPTIONS parameter for temperature, represented
in different units: Celsius, Kelvin and Fahrenheit. Arguments:

DEFAULT-VALUE : default numerical value to be used
DEFAULT-UNITS : a keyword (:C, :K or :F) denoting default units
NAME          : a string name of the resulting PARAMETER-OPTIONS
ID            : a symbolic ID of the resulting PARAMETER-OPTIONS"
  (let* ((units (reverse
                 (remove-duplicates
                  (reverse (cons default-units '(:C :K :F))))))
         (values-units (mapcar
                        (lambda (to-units)
                          (list
                           to-units
                           (convert-temperature default-units
                                                to-units
                                                default-value)))
                        units)))
    (parameter
     :name name
     :id id
     :options
     (loop for (units value) in values-units
        collect (let ((units units))
                  (parameter
                   :name (format nil "Temperature in ~A" units)
                   :id units
                   :value value
                   :units (format nil "~A" units)
                   :constructor (lambda (x) (convert-temperature units :K x))))))))
\end{minted}


This function produces a value of type \texttt{PARAMETER-OPTIONS} to specify
the temperature in convenient units. It makes sure that the provided
\texttt{DEFAULT-UNITS} will be the default choice for units. It uses
\texttt{CONVERT-TEMPERATURE} to provide the conversions between different
units:

\begin{minted}[]{common-lisp}
(defmethod convert-temperature ((from (eql :C)) (to (eql :C)) value)
  "Converts temperature VALUE from units FROM to units TO"
  value)

(defmethod convert-temperature ((from (eql :F)) (to (eql :F)) value)
  value)

(defmethod convert-temperature ((from (eql :K)) (to (eql :K)) value)
  value)

(defmethod convert-temperature ((from (eql :F)) (to (eql :C)) value)
  (* 5/9 (- value 32d0)))

(defmethod convert-temperature ((from (eql :C)) (to (eql :F)) value)
  (+ 32d0 (* 9/5 value)))

(defmethod convert-temperature ((from (eql :C)) (to (eql :K)) value)
  (+ value 273.15d0))

(defmethod convert-temperature ((from (eql :K)) (to (eql :C)) value)
  (- value 273.15d0))

(defmethod convert-temperature ((from (eql :F)) (to (eql :K)) value)
  (convert-temperature :C :K (convert-temperature :F :C value)))

(defmethod convert-temperature ((from (eql :K)) (to (eql :F)) value)
  (convert-temperature :C :F (convert-temperature :K :C value)))
\end{minted}



Arrhenius law
parameters are grouped together to form \texttt{ARRHENIUS-LAW} object and the
rest: to form \texttt{PFR-PHOSPHINE}.

Firts, Arrhenius law parameters:

\begin{minted}[]{common-lisp}
(defun make-phosphine-default-arrhenius-law ()
  (parameter
   :name "Arrhenius law for phosphine decomposition"
   :id :rate-coefficient
   :children
   (list
    (parameter
     :name "Reference rate, k0"
     :id :reference-rate-constant
     :value 10d0
     :units "1/h"
     :constructor (lambda (x) (/ x 3600d0)))
    (parameter
     :name "Activation energy, Ea"
     :id :activation-energy
     :value 160d0
     :units "kJ/mol"
     :constructor (lambda (x) (* x 1000d0)))
    (make-temperature 649d0 :C "Reference temperature, T0" :reference-temperature))
   :constructor (lambda (&rest args)
                  (apply #'make-instance 'arrhenius-law args))))
\end{minted}

In PFR model parameters, to make things a bit more interesting and
help with testing, temperature is provided as an option of temperature
in either Celcius or Fahrenheit.

\begin{minted}[]{common-lisp}
(defun make-default-pfr-phosphine ()
  (parameter
   :name "PFR for phosphine decomposition"
   :children
   (list
    (parameter
     :name "Inlet flow rate"
     :value 40d0
     :perturbation 0.2d0
     :units "mol/h"
     :constructor (lambda (x) (/ x 3600d0)))
    (parameter
     :name "Final conversion"
     :value 80d0
     :units "%"
     :constructor (lambda (x) (/ x 100d0)))
    (make-temperature 1200.2d0 :F "Operating temperature" :operating-temperature)
    (parameter
     :name "Operating pressure"
     :value 4.55d0
     :units "atm"
     :constructor (lambda (x) (* x 1.01d5)))
    (make-phosphine-default-arrhenius-law))
   :constructor (lambda (&rest args)
                  (apply #'make-instance 'PFR-phosphine args))))

(defvar *PFR-synonyms*
  '(:temperature :operating-temperature
    :ref-rate :reference-rate-constant
    :ref-temperature :reference-temperature))
\end{minted}
\end{document}