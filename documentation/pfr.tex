% Created 2016-08-23 Tue 15:27
\documentclass[listings, a4paper, fleqn, pdftex, 12pt, openany, oneside, final]{memoir}
\makeatletter
\usepackage[english]{babel}
\usepackage[T1]{fontenc}
\usepackage{amsmath}
\usepackage{amsthm}
\usepackage{latexsym}
\usepackage{amssymb}
\usepackage{amsfonts}
\usepackage{bm}
\usepackage{fourier}
\usepackage[pdftex]{graphicx}
\usepackage{subfig}
\usepackage{color}
\usepackage{natbib}
\usepackage{hyperref}%[pdftex,colorlinks=true,citecolor=blue]{hyperref}
\usepackage{lscape}
\usepackage{soul}
\usepackage{tabularx}
\usepackage{booktabs}
%\usepackage{listings}
%\usepackage{listing}
\usepackage{minted}
\chapterstyle{verville}
\beforechapskip=-17mm
\textwidth=167mm
\textheight=230mm
\voffset=-10mm
\oddsidemargin=-4mm
\evensidemargin=-4mm
%%%               Vectors                          %%%
\renewcommand{\vec}[1]{\bm{#1}}
\renewcommand{\Vec}[1]{\overline{#1}}
\newcommand{\mat}[1]{\mathbf{#1}}
\newcommand{\gmat}[1]{\vec{#1}}
%%%     Products, derivatives etc.         %%%
\newcommand{\vprod}[2]{\left[#1\times #2\right]}
\newcommand{\sprod}[2]{#1\cdot #2}
\newcommand{\der}[2]{\frac{d #1}{d #2}}
\newcommand{\Der}[2]{\frac{d}{d #2}\left(#1\right)}
\newcommand{\pder}[2]{\frac{\partial #1}{\partial #2}}
\newcommand{\Pder}[2]{\frac{\partial}{\partial #2}\left(#1\right)}
\newcommand{\dder}[2]{\frac{\delta #1}{\delta #2}}
\DeclareMathOperator{\grad}{grad}
\DeclareMathOperator{\Div}{div}
\DeclareMathOperator{\erfc}{erfc}
\DeclareMathOperator{\Expect}{E}
\DeclareMathOperator{\Var}{Var}
\newcommand{\mean}[1]{\langle #1 \rangle}
\newcommand{\const}{\mathop{\mathrm{const}}}
\newcommand{\dimension}[1]{\left[\mathrm{#1}\right]}
%%%     Project specific                         %%%
\newcommand{\Diff}{\mathcal{D}}
%%%%%%%%%%          Alexey's comments                       %%%%%%%%%%
\newcommand{\alemph}[1]{\colorbox{yellow}{#1}}
\newcommand{\red}[1]{{\color{red}#1}}
\newcommand{\alnew}[1]{%\\ul%
}
%%%%%%%%%%    Hypothesis                                          %%%%%%%%%%
\theoremstyle{definition}
\newtheorem{hypothesis}{Hypothesis}
%%%%% to avoid orphans and widows
\widowpenalty3000
\clubpenalty3000
%%% Common Lisp shortcut for minted
\newminted{cl}{}
\makeatother
\date{\today}
\title{}
\hypersetup{
 pdfauthor={},
 pdftitle={},
 pdfkeywords={},
 pdfsubject={},
 pdfcreator={Emacs 24.5.1 (Org mode 8.3.5)}, 
 pdflang={English}}
\begin{document}

\tableofcontents

\chapter{Parameter container: complete example}
\label{sec:orgheadline9}
\section{Model}
\label{sec:orgheadline6}
\subsection{Description}
\label{sec:orgheadline1}
This model is based on plug-flow reactor example (Levenspiel, 1999;
page 106). Plug flow reactor is proposed for the decomposition of
phosphine
\begin{equation*}
4\mathrm{PH}_{3g }\to \mathrm{P}_{4(g)} + 6\mathrm{H}_{2}
\end{equation*}
Denote \(A = \mathrm{PH}_{3}\), \(B = \mathrm{P}_{4}\) and \(S =
\mathrm{H}_{2}\). Reaction rate is defined using Arrhenius law
\begin{equation}
-r_{A}=kC_{A}
\end{equation}
with
\begin{equation}
k=k_{0} e^{-E/R(1/T-1/T_{0})}
\end{equation}

The volume of the reactor required to achieve the conversion \(X_{A}\) (mol
of A after reactor / mol of A before) is defined by the equation
(Levenspiel, 1999; equation 21):
\begin{equation}
V=\frac{F_{A0}}{kC_{A0}}\left[(1+\varepsilon_{A})\ln{\frac{1}{1-X_{A}}} -\varepsilon_{A}X_{A}\right]
\end{equation}
where \(F_{A0}\) is input flow rate of A and \(C_{A0}\) is initial concentration
of A. Constant \(\varepsilon_{A}=(7-4)/4=0.75\) is defined via stoichiometry
(fractional change in volume).

Concentration can be determined from ideal gas law:
\begin{equation}
C_{A0}=\frac{p_{A0}}{RT}
\end{equation}
where \(p_{A0}\) is initial partial pressure of A, \(R\) is universal gas
constant and \(T\) is operating temperature.

This model can be used to estimate the volume of the reactor given
economical constraints. A default parameters are provided as a
guidence but are the subject to change (except for constant \(R\)).

\begin{center}
\begin{tabular}{lrl}
Parameter & Default value & Units\\
\hline
\(k_{0}\) & 10 & 1/h\\
\(E\) & 160 & kJ/mol\\
\(T_{0}\) & 600 & C\\
\(T\) & 650 & C\\
\(F_{A0}\) & 40 & mol/h\\
\(p\) & 4 & atm\\
\(R\) & 8.314 & J/ mol K\\
\end{tabular}
\end{center}

\subsection{Reaction rate representation}
\label{sec:orgheadline2}
First need to represent the temperature dependence of the rate
constant

\begin{minted}[]{common-lisp}
(defconstant +universal-gas-constant+ 8.314d0
  "Universal gas constant R=8.314 J/mol K")

(defclass arrhenius-law ()
  ((reference-rate-constant
    :initarg :reference-rate-constant
    :accessor reference-rate-constant
    :documentation "Reference rate constant (k0)")
   (activation-energy
    :initarg :activation-energy
    :accessor activation-energy
    :documentation "Activation energy (Ea) in J/mol")
   (reference-temperature
    :initarg :reference-temperature
    :accessor reference-temperature
    :documentation "Reference temperature (T0) in K"))
  (:documentation
   "Arrhenius law of the reaction rate constant:
    k(T) = k0 * exp[- E/R * (1/T - 1/T0)]"))

(defmethod print-object ((object arrhenius-law) out)
  (with-slots ((k0 reference-rate-constant)
               (e activation-energy)
               (t0 reference-temperature))
      object
    (print-unreadable-object (object out :type t)
      (format out "~@<~:_k0 = ~A ~:_Ea = ~A J/mol ~:_T0 = ~A K~:>"
              k0 e t0))))

(defmethod rate-constant ((model arrhenius-law) temperature)
  (with-accessors ((k0 reference-rate-constant)
                   (e activation-energy)
                   (t0 reference-temperature))
      model
    (* k0 (exp (- (* (/ e +universal-gas-constant+) (- (/ temperature) (/ t0))))))))
\end{minted}

\subsection{Ideal gas law}
\label{sec:orgheadline3}
To find the concentration from given pressure.

\begin{minted}[]{common-lisp}
(defun ideal-gas-concentration (pressure temperature)
  "Calculates concentration (mol/m3) based in PRESSURE (Pa) and
temperature (K) using ideal gas law: c = p / RT"
  (/ pressure (* +universal-gas-constant+ temperature)))
\end{minted}

\subsection{Plug-flow reactor model for phosphine decomposition}
\label{sec:orgheadline4}
Plug flow reactor model is composed of the following parts:
\begin{itemize}
\item Inlet flow rate \(F_{A0}\),
\item Final (target) conversion \(X_{A}\),
\item Operating temperature \(T\),
\item Operating pressure \(p\), and
\item Reaction rate constant defined by Arrhenius law.
\end{itemize}

\begin{minted}[]{common-lisp}
(defclass PFR-phosphine ()
  ((inlet-flow-rate
    :initarg :inlet-flow-rate
    :documentation "F_A0 in mol/s"
    :accessor inlet-flow-rate)
   (final-conversion
    :initarg :final-conversion
    :accessor final-conversion
    :documentation "Target conversion X_A")
   (operating-temperature
    :initarg :operating-temperature
    :accessor operating-temperature
    :documentation "PFR operating temperature (K)")
   (operating-pressure
    :initarg :operating-pressure
    :accessor operating-pressure
    :documentation "PFR operating pressure (Pa)")
   (phosphine-decomposition-rate-constant
    :initarg :rate-constant
    :accessor phosphine-decomposition-rate-constant
    :documentation
    "Rate constant object specializing RATE-CONSTANT generic function"))
  (:documentation
   "Plug-flow-reactor for phosphine decomposition
    4PH3 -> P4 + 6H2"))

(defmethod print-object ((object PFR-phosphine) out)
  (with-slots ((f inlet-flow-rate)
               (x final-conversion)
               (temperature operating-temperature)
               (p operating-pressure)
               (rate-expression phosphine-decomposition-rate-constant))
      object
    (print-unreadable-object (object out :type t)
      (format out "~@<~:_F = ~A mol/s ~:_X = ~A ~:_T = ~A K ~:_p = ~A Pa~:>"
              f x temperature p))))
\end{minted}

Calculate volume of the reactor:

\begin{minted}[]{common-lisp}
(defmethod PFR-phosphine-volume ((model pfr-phosphine))
  (with-accessors ((f inlet-flow-rate)
                   (x final-conversion)
                   (temperature operating-temperature)
                   (p operating-pressure)
                   (rate phosphine-decomposition-rate-constant))
      model
    (let ((k (rate-constant rate temperature))
          (eps 0.75d0)
          (c (ideal-gas-concentration p temperature)))
      (* (/ f (* k c))
         (- (* (1+ eps) (log (/ (- 1 x))))
            (* eps x))))))
\end{minted}

\subsection{Model parameters}
\label{sec:orgheadline5}
The model consists of a large number of parameters. Arrhenius law
parameters are grouped together to form \texttt{ARRHENIUS-LAW} object and the
rest: to form \texttt{PFR-PHOSPHINE}.

Firts, Arrhenius law parameters:

\begin{minted}[]{common-lisp}
(defun make-phosphine-default-arrhenius-law ()
  (parameter
   :name "Arrhenius law for phosphine decomposition"
   :id :rate-constant
   :children
   (list
    (parameter
     :name "Reference rate, k0"
     :id :reference-rate-constant
     :value 10d0
     :units "1/h"
     :value-transformer (lambda (x) (/ x 3600d0)))
    (parameter
     :name "Activation energy, Ea"
     :id :activation-energy
     :value 160d0
     :units "kJ/mol"
     :value-transformer (lambda (x) (* x 1000d0)))
    (parameter
     :name "Reference temperature, T0"
     :id :reference-temperature
     :value 650d0
     :units "C"
     :value-transformer (lambda (x) (+ x 273.15d0))))
   :constructor (lambda (&rest args)
                  (apply #'make-instance 'arrhenius-law args))))
\end{minted}

In PFR model parameters, to make things a bit more interesting and
help with testing, temperature is provided as an option of temperature
in either Celcius or Fahrenheit.

\begin{minted}[]{common-lisp}
(defun make-default-pfr-phosphine ()
  (parameter
   :name "PFR for phosphine decomposition"
   :children
   (list
    (parameter
     :name "Inlet flow rate"
     :value 40d0
     :units "mol/h"
     :value-transformer (lambda (x) (/ x 3600d0)))
    (parameter
     :name "Final conversion"
     :value 80d0
     :units "%"
     :value-transformer (lambda (x) (/ x 100d0)))
    (parameter
     :name "Operating temperature"
     :options
     (list
      (parameter
       :name "Operating temperature (C)"
       :value 650d0
       :units "C"
       :value-transformer (lambda (x) (+ x 273.15d0)))
      (parameter
       :name "Operating temperature (F)"
       :value 1202d0
       :units "F"
       :value-transformer (lambda (x) (+ 273.15d0 (* 5/9 (- x 32d0)))))))
    (parameter
     :name "Operating pressure"
     :value 4.54d0
     :units "atm"
     :value-transformer (lambda (x) (* x 1.01d5)))
    (make-phosphine-default-arrhenius-law))
   :constructor (lambda (&rest args)
                  (apply #'make-instance 'PFR-phosphine args))))
\end{minted}

\section{Tests}
\label{sec:orgheadline7}
Use FiveAM. Put all the tests into their own suite.
\begin{minted}[]{common-lisp}
(def-suite pfr-phosphine-suite)

(in-suite pfr-phosphine-suite)
\end{minted}

Make sure \texttt{PARAMETER-VALUE} on \texttt{PARAMETER-CONTAINER} returns itself.
\begin{minted}[]{common-lisp}
(test test-parameter-container-value
  (let ((arrhenius-law (make-phosphine-default-arrhenius-law)))
    (is (eq (parameter-value arrhenius-law)
            arrhenius-law))))
\end{minted}

Instantiated object is of the right type
\begin{minted}[]{common-lisp}
(test test-parameter-container-instantiate-object
  (let ((arrhenius-law (make-phosphine-default-arrhenius-law)))
    (let ((result (instantiate-object arrhenius-law)))
      (is (typep result 'arrhenius-law)))))
\end{minted}

Container must instantiate objects recursively and the parameters
must be properly adjusted
\begin{minted}[]{common-lisp}
(test test-composite-parameter-container-instantiate-object
  (let ((pfr (make-default-pfr-phosphine)))
    ;; Checks if it propogates and converts
    (setf (parameter-value (parameter-ref pfr :inlet-flow-rate))
          50d0)
    (let ((result (instantiate-object pfr)))
      (is (typep result 'PFR-phosphine))
      (is (typep (phosphine-decomposition-rate-constant
                  result)
                 'arrhenius-law))
      (is (approx= (inlet-flow-rate result) (/ 50d0 3600d0))))))
\end{minted}

Finally, this test combines the test on parameters and the model:
compare the result with the result from Levenspiel, 1999; Example 5.5:
\begin{minted}[]{common-lisp}
(test test-pfr-volume "Compare result with Levenspiel's example"
  (let* ((pfr-parameters (make-default-pfr-phosphine))
         (pfr-model (instantiate-object pfr-parameters))
         (volume (pfr-phosphine-volume pfr-model))
         (levenspiel-result 0.148d0))
    (is (approx= levenspiel-result volume :abstol 1d-3 :reltol 1d-3))))
\end{minted}

To run the tests:
\begin{quote}
(let ((\textbf{debug-on-error} t))
  (run! 'pfr-phosphine-suite))
\end{quote}

\section{To run the interface}
\label{sec:orgheadline8}
To run the interface part, execute the following:
\begin{quote}
(parameters-interface:show-model (make-default-pfr-phosphine))
\end{quote}
\end{document}